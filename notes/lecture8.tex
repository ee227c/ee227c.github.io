
\section{Lecture 8: Conjugate gradients and Krylov subspaces}
\sectionlabel{TBD}

In this lecture, we'll develop a unified view of solving linear equations~$Ax=b$
and eigenvalue problems~$Ax=\lambda x.$ In particular, we will justify the
following picture.
%
\begin{center}
\begin{tabular}{ | c |c| c | } 
\hline
 & $Ax=b$ & $Ax=\lambda x$ \\ 
\hline
Basic & Gradient descent & Power method \\ 
\hline
Accelerated & Chebyshev iteration & Chebyshev iteration \\
\hline
Accelerated and step size free & Conjugate gradient & Lanczos \\
\hline
\end{tabular}
\end{center}

What we saw last time was the basic gradient descent method and Chebyshev
iteration for solving quadratics. Chebyshev iteration requires step sizes to be
carefully chosen. In this section, we will see how we can get a ``step-size
free'' accelerated method, known as \emph{conjugate gradient}.

What ties this all together is the notion of a Krylov subspace and its
corresponding connection to low-degree polynomials.

Our exposition follows the excellent Chapter VI
in~Trefethen-Bau~\cite{trefethen97}.

\subsection{Krylov subspaces}
%
The methods we discuss all have the property that they generate a sequence of
points iteratively that is contained in a subspace called the \emph{Kyrlov
subspace}.
%
\begin{definition}[Krylov subspace]
For a matrix $A \in \R^{n x n}$ and a vector $b \in \R^n$, 
the \emph{Krylov sequence} of order $t$ is $b, Ab, A^2b, ...., A^tb$. 
We define the \emph{Krylov subspace} as
\[
K_t(A,b)=\mathrm{span}(\{b, Ab, A^2b, \ldots, A^t\}) \subseteq \R^n\,.
\]
\end{definition}

Krylov subspace naturally connect to polynomial approximation problems.
To see this, recall that a degree~$t$ matrix polynomial is an expression of the form
$p(A)=\sum_{i=1}^t \alpha_i A^i.$ 

\begin{fact}[Polynomial connection]
The Krylov subspace satisfies
\[
K_t(A,b) = \left\{ p(A)b : \text{deg}(p) \leq t \right\}\,.
\]
\end{fact}
\begin{proof}
Note that
\begin{align*}
v \in K_t(A,b) &\Longleftrightarrow \exists \alpha_i: \ v= \alpha_0 b + \alpha_1 Ab + \cdots \alpha_t A^tb
\end{align*}
\end{proof}


From here on, suppose we have a symmetric matrix $A \in \R^{n \times n}$ that
has orthonormal eigenvectors $u_1 \ldots u_n$ and ordered eigenvalues $\lambda_1
\geq \lambda_2 \ldots \geq \lambda_n$. Recall, this means
\begin{align*}
    \langle u_i,u_j\rangle &= 0,\quad\text{for } i \neq j\\
    \langle u_i, u_i\rangle &= 1 
\end{align*}
Using that $A=\sum_i \lambda_i u_i u_i^\trans,$ it follows
\[
p(A)u_i = p(\lambda_i)u_i\,.
\]
%
Now suppose we write $b$ in the eigenbasis of $A$ as 
\[
b=\alpha_1 u_1 + ... + \alpha_n u_n
\] 
with $\alpha_i = \langle u_i,b\rangle$. It follows that
\begin{align*}
p(A)b &= \alpha_1 p(\lambda_1)u_1 + \alpha_2 p(\lambda_2) u_2 + \ldots +
\alpha_n p(\lambda_n) u_n\,.
\end{align*}
 
\subsection{Finding eigenvectors}
Given these ideas, one natural approach to finding eigenvectors is to find a
polynomial~$p$ such that
\[
p(A)b \approx \alpha_1 u_1\,.
\]
Ideally, we would have $p(\lambda_1) = 1$ and $ p(\lambda_i) = 0$ for $i > 1$,
but this is in general impossible unless we maket he degree of our polynomial as
high as the number of distinct eigenvalues of~$A.$ Keep in mind that the degree
ultimately determines the number of steps that our iterative algorithm makes.
We'd therefore like to keep it as small as possible.

That's why we'll settle for an approximate solution that has $p(\lambda_1) = 1$
and makes $\max_{i > 1} p(\lambda_i)$ as small as possible. This will give us a close
approximation to the top eigenvalue. In practice, we don't know the
value~$\lambda_1$ ahead of time. What we therefore really care about is the
ratio $p(\lambda_1)/p(\lambda_2)$ so that no matter what~$\lambda_1,$ the second
eigenvalue will get mapped to a much smaller value by~$p.$ 

We consider the following simple polynomial~$p(\lambda)=\lambda^t$ that 
satisfies
\begin{align*}
    p(\lambda_2)/p(\lambda_1) &= \left(\frac{\lambda_2}{\lambda_1}\right)^t
\end{align*}
In the case where $\lambda_1=(1+\epsilon)\lambda_2$ we need $t=O(1/\epsilon)$ 
to make the ratio small.

The next lemma turns a small ratio into an approximation result for the
top eigenvector. To state the lemma, we recall that $\tan \angle (a,b)$ is the
tangent of the angle between $a$ and $b.$

\begin{lemma}
$\tan \angle(p(A)b, u_1) \leq \max_{j>1} \frac{|p(\lambda_j)|}{|p(\lambda_1)|}
\tan \angle (b, u_1)$
\end{lemma}

\begin{proof}
We define $\theta = \angle (u_1,b)$. By this, we get
\begin{align*}
    \sin ^2 \theta &=  \sum_{j >1} \alpha_j^2 \\
    \cos ^2 \theta &= |\alpha_1|^2 \\
    \tan^2 \theta &= \sum_{j > 1} \frac{|\alpha_j^2|}{|\alpha_1|^2}
\end{align*}
Now we can write:
\begin{align*}
    \tan^2 \angle (p(A)b, u_1) = \sum_{j>1}
\frac{|p(\lambda_j)\alpha_j|^2}{|p(\lambda_1)\alpha_1|^2} \leq
\underset{j>1}{\text{max}} \frac{|p(\lambda_j)|^2}{|p(\lambda_1)|^2} \sum_{j>1} \frac{\alpha_j |^2}{|\alpha_1| ^2}
\end{align*}
We note that this last sum $ \sum_{j > 1} \frac{\alpha_j |^2}{|\alpha_1| ^2}= \tan \theta$ and we obtain our desired result.
\end{proof}
Applying the lemma to $p(\lambda) = \lambda^t$ and $\lambda_1 =
(1+\epsilon) \lambda_2$, we get  
\[
\tan \angle(p(A)b, u_1) \leq
(1+\epsilon)^{-t} \tan \angle(u_1, b)\,.
\]
If there is a big gap between
$\lambda_1$ and $\lambda_2$ this converges quickly but it can be slow if
$\lambda_1 \approx \lambda_2$. It worth noting that if we choose $b\in\R^n$ to be a
random direction, then
\[
\E\left[\tan\angle(u_1,b)\right]=O\left(\sqrt{n}\right)\,.
\]
Going one step further we can also see that the expression $p(A)b=A^tb$ can of
course be built iteratively by repeatedly multiplying by~$A.$ For reasons of
numerical stability it makes sense to normalize after each matrix-vector
multiplication. This preserved the direction of the iterate and therefore does
not change our convergence analysis. The resulting algorithms is the well known
power method, defined recursively as follows:
%
\begin{align*}
x_0 &= \frac{b}{\|b\|} \\
x_t &= \frac{A x_{t-1}}{\|Ax_{t-1}\|}
\end{align*}
%
This method goes back more than hundred years to a paper by M\"untz in 1913, but
continues to find new applications today.
%
\subsection{Applying Chebyshev polynomials}

As we would expect from the development for quadratics, we can use Chebyshev
polynomials to get a better solution the polynomial approximation problem that
we posed above. The idea is exactly the same with the small difference that we
normalize our Chebyshev polynomial slightly differently.  This time around, we
want to ensure that $p(\lambda_1) = 1$ so that we are picking out the first
eigenvalue with the correct scaling. 

%\begin{tikzpicture}[scale = 0.8]
%  %\begin{axis}[doma
%  in= 2:8,xlabel=$\lambda$,label style={font=\large},tick label style={font=\large}, ylabel style={yshift=-.1cm}, xmin=0, xmax=13, ymin=-1, ymax=1.5, xtick={}, ytick={-1, 0,1},trig format plots=rad,grid=both,grid style={dashed,gray}]
%        \begin{axis}[%
%                width=10cm,
%                height=4cm,
%                scale only axis,
%                xmin=0, xmax=7,
%                xtick={0.5, 1, 1.5,2, 2.5,3, 3.5,4, 4.5, 5, 5.5, 5.75, 6,6.5},
%                xticklabels={$\lambda_n$,,,,,,,,,,,$\lambda_1$},
%                xmajorgrids,
%                ymin=-1, ymax=1.2,
%                ymajorgrids,
%                title={$p(0) = 1$},
%                axis lines*=left,
%                line width=1.0pt,
%                mark size=2.0pt,
%                legend style={at={(1.03,1)},anchor=north west,draw=black,fill=white,align=left}];
%                \addplot[domain = 0.5:5.75, black, very thick,samples=200] {.5*(cos(3*deg(x) ))};
%                \addplot[domain = 0:2] coordinates {(0,1)(0.5,0)};
%                \addplot[domain = 6.17:8.3] coordinates {(5.75,0)(6.45,1)};
%                \addplot[red] coordinates{(0.5,1)(0.5,-1)};
%                \addplot[red] coordinates{(5.75,1)(5.75,-1)};
%              
%               
%\end{axis}
%\end{tikzpicture}
%\begin{tikzpicture}[scale = 0.8]
%        \begin{axis}[%
%                width=10cm,
%                height=4cm,
%                scale only axis,
%                xmin=0, xmax=7,
%                xtick={0.5, 1, 1.5,2, 2.5,3, 3.5,4, 4.5, 5, 5.5, 6,6.5,7},
%                xticklabels={$\lambda_n$,,,,,,,,,,,,$\lambda_1$},
%                xmajorgrids,
%                ymin=-1, ymax=1.2,
%                ymajorgrids,
%                title={$p(\lambda_1) = 1$},
%                axis lines*=left,
%                line width=1.0pt,
%                mark size=2.0pt,
%                legend style={at={(1.03,1)},anchor=north west,draw=black,fill=white,align=left}];
%                \addplot[domain = 0.5:5.75, black, very thick,samples=200] {.5*(cos(3*deg(x) ))};
%                \addplot[domain = 0:2] coordinates {(0,1)(0.5,0)};
%                \addplot[domain = 6.17:8.3] coordinates {(5.75,0)(6.45,1)};
%                \addplot[red] coordinates{(0.5,1)(0.5,-1)};
%                \addplot[red] coordinates{(6.5,1)(6.5,-1)};
%              
%               
%\end{axis}
%\end{tikzpicture}

\begin{lemma}
A suitably rescaled degree $t$ Chebyshev polynomial achieves 
\begin{equation*}
\min_{p(\lambda_1)=1} \max_{\lambda \in [\lambda_2, \lambda_n]} p(\lambda) \leq
\frac{2}{(1+\max\{\sqrt{\epsilon},\epsilon\})^t}
\end{equation*}
where $\epsilon = \frac{\lambda_1}{\lambda_2} - 1$ quantifies the gap between the first
and second eigenvalue.
\end{lemma}

Note that the bound is much better than the previous one when~$\epsilon$ is
small. In the case of quadratics, the relevant ``$\epsilon$-value'' was the inverse
condition number. For eigenvalues, this turns into the gap between the first and
second eigenvalue.

\begin{center}
\begin{tabular}{ | c |c| c | } 
\hline
 & $Ax=b$ & $Ax=\lambda x$ \\ 
\hline
$\epsilon$ & $\frac{1}{\kappa} = \frac{\alpha}{\beta}$ & $\frac{\lambda_1}{\lambda_2} -1$ \\ 
\hline
\end{tabular}
\end{center}

As we saw before, Chebyshev polynomials satisfy a recurrence relation that can
be used to derive an iterative method achieving the bound above. The main
shortcoming of this method is that it needs information about the location of
the first and second eigenvalue. Instead of describing this algorithm, we move
on to an algorithm that works without any such information.

\subsection{Conjugate gradient method}
At this point, we switch back to linear equations~$Ax=b$ for a symmetric
positive definite matrix~$A\in\R^{n\times n}.$ The method we'll see is called
\emph{conjugate gradient} and is an important algorithm for solving linear
equations. Its eigenvalue analog is the Lanczos method. While the ideas behind
these methods are similar, the case of linear equations is a bit more intuitive.
%
\begin{definition}[Conjugate gradient method]
We want to solve $Ax = b,$ with  $A\succ 0$ symmetric. The conjugate gradient
method maintains a sequence of three points:
\begin{align*}
x_0 &= 0 \tag{ ``candidate solution''} \\
r_0 &= b \tag{ ``residual''} \\
p_0 &= r_0 \tag{ ``search direction''}
\end{align*}
For $t \ge 1:$
\begin{align*}
\eta_t &= \frac{\|r_t\|}{\langle p_{t-1}, Ap_{t-1}\rangle} \tag{``step size''} \\
x_t &= x_{t-1} + \eta_t p_{t-1} \\
r_t &= r_{t-1} - \eta_t A r_{t-1} \\
p_t &= r_t + \frac{\|r_t\|^2}{\|r_{t-1}\|^2}p_{t-1}
\end{align*}
\end{definition}

\begin{lemma}
\lemmalabel{lecture8-conjugategradient}
The following three equations must always be true for the conjugate gradient
method algorithm:
\begin{itemize}
\item $\mathrm{span}(\{r_0, ...r_{t-1}\}) = K_t(A,b)$
\item For $j<t$ we have $\langle r_t, r_j\rangle = 0$ and in particular
$r_t \perp K_t(A,b).$
\item The search directions are conjugate $p_i^{\trans} A p_j = 0$ for $i\ne j.$
\end{itemize}
\end{lemma}
\begin{proof}
Proof by induction (see Trefethen and Bau). Show that the conditions are true
initially and stay true when the update rule is applied.  
\end{proof}

\begin{lemma}
Let $\|u\|_A = \sqrt{u^\trans A u}$ and $\langle u,v\rangle_A = u^\trans A v$ and 
$e_t = x^* - x_t$. Then $e_t$ minimizes $\|x^* - x\|_A$ over all vectors $x \in K_{t-1}$.
\end{lemma}

\begin{proof}
We know that $x_t \in K_t$. Let $x \in K_t$ and define $x = x_t - \delta$. 
Then, $e = x^* - x = e_t + \delta.$
We compute the error in the $A$ norm:
\begin{align*}
\|x^* - x\|_A^2 
&= (e_t + \delta)^{\trans} A(e_t + \delta) \\
&= e_t^{\trans}Ae_t + \delta^{\trans}A\delta + 2e_t^{\trans} A \delta\,
\end{align*}
Bby definition~$e_t^{\trans}A = r_t$.
Note that $\delta \in K_t.$ 
By \lemmaref{lecture8-conjugategradient}, we have that $r_t \perp K_t(A,b).$
Therefore, $2e_t^{\trans} A \delta = 0$ and hence, 
\begin{align*}
\|e\|_A^2
= \|x^* - x\|_A^2 
= e_t^{\trans}Ae_t + \delta^{\trans}A\delta
\ge \|e_t\|_A\,.
\end{align*}
In the last step we used that $A\succ0.$
\end{proof}

What the lemma shows, in essence, is that conjugate gradient solves the
polynomial approximation problem:
\[
\min_{p\colon\deg(p)\le t, p(0)=1} \|p(A)e_0\|_A\,.
\]
Moreover, it's not hard to show that
\[
\min_{p\colon\deg(p)\le t, p(0)=1} 
\frac{\|p(A)e_0\|_A}{\|e_0\|_A}
\le 
\min_{p\colon\deg(p)\le t, p(0)=1} 
\max_{\lambda\in\Lambda(A)}\left| p(\lambda)\right|\,.
\]
In other words, the error achieved by conjugate gradient is no worse that the
error of the polynomial approximation on the RHS, which was solved by the
Chebyshev approximation. From here it follows that conjugate gradient must
converge at least as fast in $\|\cdot\|_A$-norm than Chebyshev iteration.

