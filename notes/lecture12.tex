\section{Coordinate descent}
There are many classes of functions for which it is very cheap 
to compute directional derivatives along the standard basis vectors
$e_i, i \in [n]$.
%
For example, 
%
\begin{eqnarray}
f(x) = \|x\|^2\quad \text{ or }\quad f(x) = \|x\|_1
\end{eqnarray}
%
This is especially true of common regularizers, 
%
which often take the form 
\begin{eqnarray}
R(x) = \sum_{i=1}^n R_i(x_i) \ .
\end{eqnarray}
%
More generally, many objectives and regularizes exhibit ``group sparsity''; that is,
%
\begin{eqnarray}
R(x) =  \sum_{j=1}^m R_{j}(x_{S_j})
\end{eqnarray}
where each $S_j, j \in [m]$ is a subsect of $[n]$, and similarly for $f(x)$.
%
Examples of functions with block decompositions and group sparsity include:
\begin{enumerate} 
	\item Group sparsity penalties;
	\item Regularizes of the form $R(U^\top x)$, where $R$ is
    coordinate-separable, and $U$ has sparse columns and so
    $(U^\top x) = u_i^\top x$ depends only on the nonzero entries of $U_i$;
	\item Neural networks, where the gradients with respect to some weights can be
    computed ``locally''; and
	\item ERM problems of the form 
    \begin{eqnarray}
    f(x) := \sum_{i=1}^n \phi_i(\langle w^{(i)} , x \rangle )
    \end{eqnarray}
    where $\phi_i: \R \to \R$, and $w^{(i)}$ is zero except in a few coordinates. 
\end{enumerate} 

% Code the produce the figure below
\begin{figure}[t]
\centering
\includegraphics{figures/lecture12-function_coordinate_graph.pdf}

%\definecolor{myblue}{RGB}{80,80,160}
%\definecolor{mygreen}{RGB}{80,160,80}
%\usetikzlibrary{positioning,chains,fit,shapes,calc}
%
%\begin{tikzpicture}[thick,
%  every node/.style={draw,circle},
%  fsnode/.style={fill=myblue},
%  ssnode/.style={fill=mygreen},
%  every fit/.style={draw=none},
%  shorten >= 3pt,shorten <= 3pt
%]
%
%% Function
%\begin{scope}[start chain=going right,node distance=7mm]
%\foreach \i in {1,2,...,5}
%  \node[fsnode,on chain] (x\i) [label=below: $x_\i$] {};
%\end{scope}
%\node [myblue,fit=(x1) (x5),label=left:variable] {};
%
%% Coordinates
%\begin{scope}[yshift=2cm,xshift=.5cm,start chain=going right,node distance=7mm]
%\foreach \i in {1,2,...,4}
%  \node[ssnode,on chain] (f\i) [label=above: $f_\i$] {};
%\end{scope}
%\node [mygreen,fit=(f1) (f4),label=left:function] {};
%
%% Edges
%\draw (x1) -- (f1);
%\draw (x2) -- (f2);
%\draw (x3) -- (f4);
%\draw (x4) -- (f4);
%\draw (x5) -- (f1);
%\draw (x5) -- (f3);
%\end{tikzpicture}
\vspace{-20pt}
\caption{
  Example of the bipartite graph between component functions
  $f_i, i \in [m]$ and variables $x_j, j \in [n]$ induced by the
  group sparsity structure of a function $f : \R^n \to \R^m$.
  An edge between $f_i$ and $x_j$ conveys that the $i$th component function
  depends on the $j$th coordinate of the input.
}
\end{figure}

\subsection{Coordinate descent}
  Denote $\partial_i f= \frac{\partial f}{x_i}$.
  For each round $t = 1,2,\dots$, the coordinate descent algorithm
  chooses an index $i_t \in [n]$, and computes
	\begin{eqnarray}
	x_{t+1} = x_t - \eta_t\partial_{i_t}f(x_t) \cdot e_{i_t} \ .
	\end{eqnarray}
  This algorithm is a special case of stochastic gradient descent. For
	\begin{eqnarray}
    \E[x_{t+1} | x_t]
    &=& x_t - \eta_t \E [\partial_{i_t}f(x_t) \cdot e_{i_t}] \\
    &=& x_t - \frac{\eta_t}{n} \sum_{i=1}^n \partial_{i}f(x_t) \cdot e_i \\
    &=& x_t - \eta_t \nabla f(x_t) \ .
	\end{eqnarray}
	Recall the bound for SGD: If $\E[g_t] = \nabla f(x_t)$, then SGD with step size $\eta = \frac{1}{BR}$ satisfies
	\begin{eqnarray}
	\E[f(\frac{1}{T}\sum_{t=1}^T x_t)] - \min_{x \in \Omega}f(x) \le \frac{2 BR}{\sqrt{T}}
	\end{eqnarray}
	where $R^2$ is given by $\max_{x \in \Omega} \|x-x_1\|^2_2$ and $B = \max_{t}\E[\|g_t\|^2_2]$. In particular, if we set $g_t = n \partial_{x_{i_t}}f(x_t) \cdot e_{i_t}$, we compute that
	\begin{eqnarray}
    \E[\|g_t\|^2_2]
    = \frac{1}{n} \sum_{i=1}^n \|n\cdot \partial_{x_{i}}f(x_t) \cdot e_i\|^2_2
    = n \|\nabla f(x_t)\|^2_2 \ .
	\end{eqnarray}
	If we assume that $f$ is $L$-Lipschitz, we additionally have that 
	$\E[\|g_t\|^2] \le nL^2$.
  This implies the first result:
	\begin{proposition} Let $f$ be convex and $L$-Lipschitz on $\R^n$.
    Then coordinate descent with step size $\eta = \frac{1}{n R}$ has convergence rate 
	\begin{eqnarray}
	\E[f(\frac{1}{T}\sum_{t=1}^T x_t)] - \min_{x \in \Omega}f(x) \le 2LR\sqrt{n/T}
	\end{eqnarray}
	\end{proposition}
	\subsection{Importance sampling}
	In the above, we decided on using the uniform distribution to sample a
  coordinate. But suppose we have more fine-grained information. In particular, what if we knew that we could bound $\sup_{x \in \Omega} \|\nabla f(x)_i\|_2 \le L_i$? An alternative might be to sample in a way to take $L_i$ into account. This motivates the ``importance sampled'' estimator of $\nabla f(x)$, given by
	\begin{eqnarray}
	g_t = \frac{1}{p_{i_t}} \cdot \partial_{i_t}f(x_{t}) \text{ where } i_t \sim
    \mathrm{Cat}(p_1,\dots,p_n) \ .
	\end{eqnarray}
	Note then that $\E[g_t] = \nabla f(x_t)$, but
	\begin{eqnarray}
    \E[\|g_t\|^2_2]
    &=& \sum_{i=1}^n (\partial_{i_t}f(x_{t}))^2/p_i^2\\
    &\le& \sum_{i=1}^n L_i^2/p_i^2
	\end{eqnarray}
	In this case, we can get rates 
	\begin{eqnarray}
	  \E[f(\frac{1}{T}\sum_{t=1}^T x_t)] - \min_{x \in \Omega}f(x)
    \le 2R\sqrt{1/T}\cdot \sqrt{\sum_{i=1}^n L_i^2/p_i^2}
	\end{eqnarray}
	In many cases, if the values of $L_i$ are heterogenous, we can optimize the values of $p_i$. 
	\subsection{Importance sampling for smooth coordinate descent}
	In this section, we consider coordinate descent with a \emph{biased}
  estimator of the gradient. Suppose that we have, for $x \in \R^n$
  and $\alpha \in \R$, the inequality
	\begin{eqnarray}
    |\partial_{x_i} f(x) - \partial_{x_i} f(x + \alpha e_i)|
    \le \beta_{i}|\alpha|
	\end{eqnarray}
	where $\beta_i$ are possibly heterogenous. Note that if that $f$ is twice-continuously differentiable, the above condition is equivalent to $\nabla^2_{ii}f(x) \le \beta_i$, or $\mathrm{Diag}(\nabla^2 f(x)) \preceq \mathrm{diag}(\boldsymbol{\beta})$.  Define the distribution $p^\gamma$ via
	\begin{eqnarray}
	p^{\gamma}_i = \frac{\beta_i^{\gamma}}{\sum_{j=1}^n \beta_j^{\gamma}}
	\end{eqnarray}
	We consider gradient descent with the rule called $\mathrm{RCD}(\gamma)$
	\begin{eqnarray}\label{RCDgamma}
    x_{t+1} = x_t - \frac{1}{\beta_{i_t}} \cdot \partial_{i_t}(x_t) \cdot e_{i_t}, \text{ where } i_t \sim p^{\gamma}
	\end{eqnarray}
  Note that as $\gamma \to \infty$, coordinates with larger values of $\beta_i$
  will be selected more often.
	Also note that this is \emph{not generally} equivalent to SGD, because 
	\begin{eqnarray}
    \E\left[\frac{1}{\beta_{i_t}}\partial_{i_t}(x_t)e_i\right]
    = \frac{1}{\sum_{j=1}^n \beta_j^{\gamma}} \cdot \sum_{i=1}^n \beta_{i}^{\gamma - 1} \partial_i f(x_t)e_i
    = \frac{1}{\sum_{j=1}^n \beta_j^{\gamma}} \cdot \nabla f(x_t) \circ (\beta_i^{\gamma - 1})_{i \in [n]}
	\end{eqnarray}
	which is only a scaled version of $\nabla f(x_t)$ when $\gamma = 1$. Still, we can prove the following theorem:
	\begin{theorem}
  \label{thm:6.7}
  Define the weighted norms
	\begin{eqnarray}
	\|x\|_{[\gamma]}^2 := \sum_{i=1}^n x_i^2 \beta_i^\gamma \text{ and } \|x\|_{[\gamma]}^{*2} := \sum_{i=1}^n x_i^2 \beta_i^{-\gamma}
	\end{eqnarray}
	and note that the norms are dual to one another.
  We then have that the rule $\mathrm{RCD}(\gamma)$ produces iterates satisfying
	\begin{eqnarray}
	  \E[f(x_t) - \arg\min_{x \in \R^{n}} f(x)]
    \le \frac{2 R^2_{1-\gamma} \cdot \sum_{i=1}^n \beta_{i}^{\gamma}}{t-1}~,
	\end{eqnarray}
	where $R^2_{1-\gamma} = \sup_{x \in \R^n:f(x) \le f(x_1)} \|x - x^*\|_{[1 - \gamma]}$.
	\end{theorem}
	\begin{proof}
	Recall the inequality that for a general $\beta_g$-smooth convex function $g$, one has that
	\begin{eqnarray}
    g\left(u - \frac{1}{\beta_g}\nabla g(u)\right) - g(u)
    \le -\frac{1}{2\beta_g} \|\nabla g\|^2
	\end{eqnarray}
	Hence, considering the functions $g_i(u;x) = f(x + ue_i)$, we see that $\partial_i f(x) = g_i'(u;x)$, and thus $g_i$ is $\beta_i$ smooth. Hence, we have 
	\begin{eqnarray}
    f\left(x - \frac{1}{\beta_i}\nabla f(x)e_i\right) - f(x)
    = g_i(0 - \frac{1}{\beta_g}g_i'(0;x)) - g(0;x)
    \le -\frac{g_i'(u;x)^2}{2\beta_i} = - \frac{\partial_i f(x)^2}{2\beta_i} \ .
	\end{eqnarray}
	Hence, if $i~p^\gamma$, we have
	\begin{eqnarray}
	  \E[f(x - \frac{1}{\beta_i}\partial_i f(x)e_i) - f(x)]
    &\le& \sum_{i=1}^n p_i^\gamma \cdot - \frac{\partial_i f(x)^2}{2\beta_i}\\
	  &=& -\frac{1}{2\sum_{i=1}^n \beta_i^{\gamma}} \sum_{i=1}^n \beta^{\gamma - 1}\partial_i f(x)^2 \\
	  &=& -\frac{\|\nabla f(x)\|^{*2}_{[1-\gamma]})}{2\sum_{i=1}^n \beta_i^{\gamma}}
	\end{eqnarray}
	Hence, if we define $\delta_t = \E[f(x_t) - f(x^*)]$, we have that
	\begin{eqnarray}
	\delta_{t+1} - \delta_t \le -\frac{\|\nabla f(x_t)\|^{*2}_{[1-\gamma]}}{2\sum_{i=1}^n \beta_i^{\gamma}} 
	\end{eqnarray}
	Moreover, with probability $1$, one also has that $f(x_{t+1}) \le f(x_t)$, by the above. We now continue with the regular proof of smooth gradient descent. Note that
	\begin{eqnarray*}
	\delta_t &\le& \nabla f(x_t)^\top(x_t - x_*)\\
	&\le& \|\nabla f(x_t)\|_{[1-\gamma]}^*\|x_t - x_*\|_{[1-\gamma]}\\
	&\le& R_{1-\gamma}\|\nabla f(x_t)\|_{[1-\gamma]}^*\ .
	\end{eqnarray*}
	Putting these things together implies that
	\begin{eqnarray}
	\delta_{t+1} - \delta_t \le -\frac{\delta_t^2}{2R_{1-\gamma}^2\sum_{i=1}^n \beta_i^{\gamma}} 
	\end{eqnarray}
	Recall that this was the recursion we used to prove convergence in the non-stochastic case.
	\end{proof}
	\begin{theorem} If $f$ is in addition $\alpha$-strongly convex w.r.t to $\|\cdot\|_{[1-\gamma]}$, then we get 
	\begin{eqnarray}
    \E[f(x_{t+1}) - \arg\min_{x \in \R^{n}} f(x)]
    \le \left(1 - \frac{\alpha}{\sum_{i=1}^n \beta_i^\gamma}\right)^t (f(x_1) - f(x^*)) \ .
	\end{eqnarray}
	\end{theorem}
	\begin{proof} We need the following lemma:
  \begin{lemma}
    \label{lemma:6.9} 
    Let $f$ be an $\alpha$-strongly convex function w.r.t to a norm
    $\|\cdot\|$. Then, $f(x) - f(x^*) \le \frac{1}{2\alpha} \|\nabla
    f(x)\|_*^2$\ .
	 \end{lemma}
	 \begin{proof}
	 \begin{eqnarray*}
     f(x) - f(y)
     &\le& \nabla f(x)^\top (x -y ) - \frac{\alpha}{2}\|x - y\|^2_2\\
	   &\le& \|\nabla f(x)\|_* \|x - y\|^2 - \frac{\alpha}{2}\|x - y\|^2_2\\
	   &\le& \max_t \|\nabla f(x)\|_* t - \frac{\alpha}{2}t^2\\
	 	 &=&  \frac{1}{2\alpha} \|\nabla f(x)\|_*^2 \ .
	 \end{eqnarray*}
	 \end{proof}

   Lemma \ref{lemma:6.9} shows that 
	 \begin{eqnarray*}
     \|\nabla f(x_s)\|^{*2}_{[1-\gamma]}
     \ge 2 \alpha \delta_s \ .
	 \end{eqnarray*}
   On the other hand, Theorem \ref{thm:6.7} showed that
   \begin{eqnarray}
   \delta_{t+1} - \delta_t \le -\frac{\|\nabla f(x_t)\|^{*2}_{[1-\gamma]}}{2\sum_{i=1}^n \beta_i^{\gamma}} 
   \end{eqnarray}
   Combining these two, we get
	 \begin{eqnarray}
     \delta_{t+1} - \delta_t &\le& -\frac{\alpha \delta_t}{\sum_{i=1}^n \beta_i^{\gamma}}  \\
     \delta_{t+1} &\le&  \delta_t \left(1  -\frac{\alpha}{\sum_{i=1}^n \beta_i^{\gamma}} \right)~.
	 \end{eqnarray}
   Applying the above inequality recursively and recalling that
   $\delta_t = \E[f(x_t) - f(x^*)]$ gives the result.

	\end{proof}

	\subsection{Random coordinate vs.~stochastic gradient descent}
	What's surprising is that $\mathrm{RCD}(\gamma)$ is a descent method, despite being random. This is not true of normal SGD. 
	But when does $\mathrm{RCD}(\gamma)$ actually do better? If $\gamma = 1$, the savings are proportional to the ratio of $\sum_{i=1} \beta_i / \beta \cdot (T_{coord}/T_{grad})$. When $f$ is twice differentiable, this is the ratio of 
	\begin{eqnarray}
	\frac{\tr(\max_{x} \nabla^2 f(x))}{\|\max_{x} \nabla^2 f(x)\|_{\op}} (T_{coord}/T_{grad})
	\end{eqnarray}
	\subsection{Other extensions to coordinate descent}
	\begin{enumerate}
		\item Non-Stochastic, Cyclic SGD
		\item Sampling with Replacement
		\item Strongly Convex + Smooth!?
		\item Strongly Convex (generalize SGD)
		\item Acceleration? See \cite{tu2017breaking}
	\end{enumerate}

