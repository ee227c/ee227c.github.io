%\documentclass[12pt]{article}
%
%\usepackage{macros}
%
%\title{Course Notes for EE227C (Spring 2018):\\
 Convex Optimization and Approximation }
\author{Instructor: Moritz Hardt\\
{\small Email: \tt hardt+ee227c@berkeley.edu}\\ ~\\
Graduate Instructor: Max Simchowitz\\
{\small Email: \tt msimchow+ee227c@berkeley.edu}\\ ~\\
}

%
%\begin{document}
%	
%	\maketitle
%	
\section{Submodular functions and Lov\'{a}sz extension}

\begin{definition}[Submodular functions]
	Let $N = \{1, \dots, n\}$, and $2^N$ denote the power set of $N$.
	Then a function $f : 2^N \rightarrow \mathbb{R}$ is \emph{submodular} if:
	\begin{align*}
	f(A \cup \{ j \} ) - f(A) \geq f(B \cup \{ j \}) - f(B)
	\end{align*}
	for all $A \subseteq B \subseteq N$ and for all $j \in N$.
	
	This is equivalent to:
	\begin{align*}
	f(A \cap B) + f(A \cup B) \leq f(A) + f(B)
	\end{align*}
	for all $\forall A, B \subseteq N$.
	
\end{definition}

Some examples of submodular functions:
\begin{itemize}
	\item Let $A$ be a subset of nodes of a bipartite graph. Then $f(A) = | \text{neighborhood}(A) |$ is submodular.
	
	\item Let $X = \{ X_1, \dots X_n \}$ be a set of random variables, and $A \subseteq X$. Then $f(A) = H(A)$, where $H$ denotes entropy, is submodular.
	
	\item Let $E, V$ be the edges and nodes of a graph, and $A \subseteq V$. Then the size of the graph cut, $f(A) = \big| \{ (u, v) \in E : u \in A, v \in V \setminus A \} \big| $, is submodular.
\end{itemize}

There is an equivalence between submodular and boolean functions: a submodular function $f: 2^N \rightarrow \mathbb{R}$ can be expressed as a boolean function $f: \{0, 1\}^n \rightarrow \mathbb{R}$.

\begin{definition}[Lov\'{a}sz extension]
Let $f(x)$ be a submodular function. Then its Lov\'{a}sz extension, $\hat f(z): [0, 1]^n \rightarrow \mathbb{R}$ is given by:
\begin{align*}
\hat f(z) = \mathbb{E}_{\lambda \sim \text{U}(0, 1)} \bigg[ f(\{i : z_i \geq \lambda \}) \bigg]
\end{align*}
where $\text{U}(0, 1)$ denotes the uniform distribution on $[0, 1]$.
\end{definition}

Some observations about $\hat f(z)$:
\begin{itemize}
	\item For any $x \in \{0, 1\}^n$, $\hat f(x) = f(x)$. That is, $f$ and $\hat f$ agree on the domain of $f$.
	
	\item For any $z \in [0, 1]^n$, there exists an $x \in \{0, 1\}^n$, such that $f(x) \leq \hat f(z)$. Also, $\min_{x \in \{0, 1\}^n} f(x) = \min_{z \in [0,1]^n} \hat f(z)$.
	 
\end{itemize}


\begin{theorem}[Submodularity and convexity]
\theoremlabel{submod_cvx}
Consider a function $f(x) : 2^N \rightarrow \mathbb{R}$ and its Lov\'{a}sz extension $\hat f(z) : 2^N \rightarrow \mathbb{R}$. Then $f(x)$ is submodular iff $\hat f(z)$ is convex.

\end{theorem}
Proof. We only show the forward direction ($f$ is submodular $\Rightarrow$ $\hat f$ is convex), the reverse direction is left as an exercise.

Without loss of generality, assume $z$ is order: $z_1 \geq z_2 \geq \dots \geq z_n$.

Also, let $S_i = \{1, 2, \dots, i\}$. We also take $f(\emptyset) = 0$.

Then:
\begin{align*}
\hat f(z) 
&= \sum_{i=1}^{n-1} P(z_{i+1} \leq \lambda \leq z_i) \cdot f(S_i) + z_n \cdot f(S_n) \\
&= \sum_{i=1}^{n-1} (z_{i+1} - z_i) \cdot f(S_i) + z_n \cdot f(S_n) \\
\end{align*}

Now, consider the following lemma:
\begin{lemma}
\lemmalabel{lovaszLP}
The Lov\'{a}sz extension $\hat f(z)$ of submodular $f(x)$ can be expressed an LP:
\begin{align*}
\hat f(z) = \max_x x^T z : x(S) \leq f(S), x(N) = f(N), \forall S \subseteq N
\end{align*}
\end{lemma}
where $x(S) = \sum_{i \in S} x_i$.  Let $F$ denote the feasible region.

Given \lemmaref{lovaszLP}, the rest of the proof follows easily:
\begin{align*}
\hat f(\lambda z + (1 - \lambda) z') 
&= \max_{x \in F} x^T (\lambda z + (1-\lambda)z') \\
&\leq \lambda \max_{x \in F} x^T z + (1 - \lambda) \max_{x  \in F} x^T z' \\
&= \lambda \hat f(z) + (1 - \lambda) \hat f(z')
\end{align*}



Now we just have to prove \lemmaref{lovaszLP}.

Consider the dual problem to the above LP:
\begin{align*}
\min_y & \sum_{S \subseteq N} y_S \cdot f(S) : \sum_{S \subseteq N} y_S e_S = z, y_S \geq 0 \\
& \text{where } (e_S)_i  = \begin{cases} i & \text{ if $i \in S$ } \\ 0 & \text{ else} \end{cases}
\end{align*}

Our goal is to find a primal-dual feasible solution $x^*, y^*$ that satisfies:
\begin{align*}
\hat f(z) = c^T x^* = \sum_{S \subseteq N} y_S^* \cdot f(S)
\end{align*}

Consider the following $x^*, y^*$:
\begin{align*}
x_i^* &= f(S_i) - f(S_{i-1}) \\
y_S^* &= \begin{cases}
z_i - z_{i+1} & \text{if } S = S_i \\
z_n & \text{if } S = N \\
0 & \text{else}
\end{cases}
\end{align*}

First, we show that $x^*$ is primal feasible.

 For the constraint $x(N) = f(N)$, we see that:
\begin{align*}
x^*(N) = \sum_{i=1}^{n} x_i^*
&= \sum_{i=1}^{n} f(S_i) - f(S_{i-1}) \\
&= f(N) - f(\emptyset) = f(N)
\end{align*}

For the other constraint, $x(S) \leq f(S)$, we proceed by induction on $|S|$ (the size of $S$).

As a base case, we take $S = \emptyset$.

Suppose that the constraint is satisfied for all sets of size $i - 1$.

Then consider an arbitrary $S$, whose largest element is $i$:
\begin{align*}
f(S) + f(S_{i-1})
&\geq f(S \cup S_{i-1}) + f(S \cup S_{i-1}) \\
&= f(S_i) + f(S \setminus \{i\}) \\
&\geq f(S_i) + x^*(S \setminus \{i\}) \\
\Rightarrow f(S) 
&\geq f(S_i) - f(S_{i-1}) + x^*(S \setminus \{i\}) \\
&= x_i^* + x^*(S \setminus \{i\}) \\
&= x^*(S)
\end{align*}
where we used fact that $f$ is submodular and that $i$ is the largest element in $S$.

It follows that $x^*(S) \leq f(S)$, for all $S \subseteq N$, and thus $x^*$ is primal feasible.

Next, we show that $y^*$ is dual feasible.
\begin{align*}
\sum_{S \subseteq N} y_S^* \cdot f(S)  
&= \sum_{j = 1}^{n-1} (z_j - z_{j+1}) e_{S_j} + z_n e_n \\
\bigg( \sum_{S \subseteq N} y_S^* \cdot f(S)  \bigg)_i
&= \bigg( \sum_{j = 1}^{n-1} (z_j - z_{j+1}) e_{S_j} + z_n e_n \bigg)_i & \text{ (consider it elementwise) } \\
&= \sum_{j = i}^{n} (z_j - z_{j+1}) + z_n \\
&= z_i
\end{align*}
Since this equality holds elementwise for each $i$, we see that  $\sum_{S \subseteq N} y_S^* \cdot f(S)  = z$.

We also observe that $y_S^* \geq 0$ due to the nonincreasing ordering of $z_1, \dots, z_n$.

Thus, it follows that $y_S^*$ is dual feasible.

Finally, we show that $x^*, y^*$  are optimal, achieving a dualty gap of zero:
\begin{align*}
z^T x^*
&= \sum_{i=1}^{n} z_i \big( f(S_i) - f(S_{i-1}) \big) \\
&= \sum_{i=1}^{n} (z_i - z_{i+1}) f(S_i) + z_n f(S_n) \\
&= \hat f(z) \\
\hat f(z) 
&= \sum_{i=1}^{n} (z_i - z_{i+1}) f(S_i) + z_n f(S_n) \\
&= \sum_{S \subseteq N} y_S^* f(S)
\end{align*}

This concludes the proof of \lemmaref{lovaszLP}, and that of \theoremref{submod_cvx}.

%  \end{document}